\chapter{Einleitung}
IT-Sicherheit wird in der Software Entwicklung ein immer wichtigeres  Thema. In Abbildung \ref{FIG:statistic-ausgaben-it-sicherheit} ist der Anstieg der Ausgaben für IT-Sicherheit zu sehen. Dieser ist in den vergangen Jahren kontinuierlich gestiegen und auch in der Zukunft sollen die Ausgaben im Jahr 2025 im Vergleich zu 2020 um 57.5\% steigen. Das zeigt das vielen Unternehmen die IT-Sicherheit wichtiger wird. Die Kosten sollen aber natürlich trotzdem immer möglichst gering bleiben.
DevOps ist mittlerweile ein verbreitete und viel genutzte Entwicklungsmethode (siehe Abb. \ref{FIG:devops-important}) um den Entwicklungsprozess zu beschleunigen und somit auch Kosten zu sparen. Um den It-Sicherheitsaspekt einfach in die Softwareentwicklung zu integrieren wurde so DevSecOps erschaffen. Der Prozess vereint DevOps mit Sicherheitsverfahren. \cite{security-model}
Zwei der größten Plattformen für DevOps sind GitHub und Azure DevOps. GitHub liefert GitHub Actions und Azure DevOps bietet Azure Pipelines um Operationen aus dem DevOps Prozess zu automatisieren. Diese Frage ist welches Tool den DevSecOps Prozess am besten integriert um sichere Software zu erstellen.


\begin{figure}[H]
	{\caption{Ausgaben für IT-Sicheheit in Deutschland in den Jahren 2017 bis 2020 und Prognose bis 2025 (in Milliarden Euro)}
		\label{FIG:statistic-ausgaben-it-sicherheit}}
	{\includegraphics[width=1\textwidth]{figures/statistic-ausgaben-it-sicherheit.png}}
\end{figure}

\chapter{Was bedeutet DevSecOps}
\section{Zusammenfassung DevOps}
DevOps setzt sich aus den Begriffen \glqq Development \grqq{}(Entwicklung) und \glqq Operations \grqq{}(Vorgänge) zusammen. Die traditionelle Trennung von Entwicklung und Betrieb führt oft zu Interessenskonflikten. Entwickler wollen stetig die Software verbessern, der Betrieb hingegen will Änderungen vermeiden um die Stabilität des System zu gewährleisten. \cite{git-ops}  DevOps soll das Entwickeln und den Betrieb von Software näher zusammenführen. Es entsteht eine Softwareentwicklungs-Prozess in der man durch Automatisierungen das Bauen, Testen und Bereitstellen von Software beschleunigen will. Dies erreicht man durch Praktiken wie Continous Integration, Continous Delivery, Continous Deployment, automatisiertes Teste, Infrastructure-as-Code und automatische Veröffentlichungen. Durch die automatische Bereitstellung der Anwendung und der Infrastruktur entsteht eine schnellere Bereitstellung, bessere Softwarequalität und auch schon mehr IT-Sicherheit. \cite{dev-ops} Außerdem ist DevOps auch offene Zusammenarbeit, Kommunikation, Transparenz, Eingestehen von Fehlern und das gemeinsame lösen von Problemen, um Konflikte im Team zu vermeiden. Man will schnelles Feedback ermöglichen und somit das Risiko der Softwareentwicklung minimieren. DevOps beschränkt sich also nicht nur auf technische Hilfsmittel sondern bietet eine Kultur um den Entwicklungsprozess immer weiter zu verbessern.\cite{git-ops}

\section{Definition von DevSecOps}
Der Begriff DevSecOps baut auf den Prinzipien und Praktiken von DevOps auf und fügt den Sicherheitsaspekt in der Entwicklung noch weiter in den Vordergrund. Durch DevSecOps soll IT-Security schon vom Start eines Projektes mitgeplant und auch in Bereitstellungsprozess integriert werden.

\section{DevSecOps Kultur}
Interessant ist bei DevOps sowie DevSecOps, das es sich um sehr offene Begriffe handelt. 

\chapter{DevSecOps mit Github Actions}

\chapter{DevSecOps mit Azure DevOps}

\chapter{Vergleich von Github Actions und Azure DevOps}

\chapter{Fazit}
Github ist am Aufstreben blablabla